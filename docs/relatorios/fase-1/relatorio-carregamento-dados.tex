\documentclass[12pt,oneside,a4paper]{article}

\usepackage[utf8]{inputenc}
\usepackage[document]{ragged2e}

\usepackage{times} % Use the Times New Roman font

\usepackage[brazilian]{babel}
\usepackage[margin=1in]{geometry}
\usepackage{natbib}
\usepackage{graphicx}
\usepackage{float}
\usepackage{amsmath}
\usepackage{setspace}

\begin{document}

\begin{titlepage}
\enlargethispage{\footskip}
\setlength{\parindent}{0pt}

\begin{center}
    \onehalfspacing
    \Large Universidade Federal de Minas Gerais
    \\ Departamento de Ciência da Computação
    
    \vspace{80mm}
    
    \Large{DCC057 Mineração de Dados}
    \\ \Large \textbf{Trabalho Prático 3 --- Relatório de Carregamento dos Dados}
    
\end{center}

\vspace{\fill}

\begin{minipage}[b]{\textwidth}
    \centering
    \onehalfspacing
    \large
    Alexander Thomas Mol Holmquist \\
    15 de março de 2021

\end{minipage}

\end{titlepage}

\justify

\section{Objetivo}

Este relatório tem como objetivo delinear conclusões a respeito dos dados, obtidas durante o processo básico de carregamento.

\section{Localização e Divisão dos Dados}

O diretório principal para dados é "data". Os dados são compostos de 3297 imagens, e estão divididos fisicamente em dois diretórios, "train" e "test". Esta divisão se refere a uma separação aleatória de 5 dobras realizada previamente sobre os dados. Assim, o conjunto de treinamento agrupa 2637 imagens, enquanto o conjunto de testes agrupa 660 imagens.

Um nível além, há mais uma divisão, desta vez entre "benign" e "malignant". Como os nomes indicam, as imagens rotuladas como sendo de câncer benigno foram colocadas em "benign", e as rotuladas como maligno foram colocadas em "malignant".

\section{Codificação dos Arquivos}

Os arquivos das imagens estão todos no formato JPEG. Cada imagem possui dimensões 224 x 224 pixels, sendo que cada pixel é um valor RGB. O espaço total ocupado pelos arquivos é de aproximadamente 171.7MB.

Este formato pode ser carregado em python3 através da biblioteca skimage, disponível publicamente. Especificamente, a função skimage.io.imread lê o arquivo referido pelo caminho passado como argumento, e retorna um vetor Numpy de dimensões (224, 224, 3).

\section{Manipulações}
\label{sec:manip}

Um pouco de prática com os dados de treinamento leva a um entendimento de que são muito extensos para se utilizar em qualquer algoritmo, e a sua dimensionalidade é muito grande, quando transformados em uma matriz de dados (dimensões 2697, 150528 para dados de treino).

Os algoritmos padrão de redução de dimensionalidade: PCA e KernelPCA foram aplicados, mas não terminam. Ficou evidente, então, a necessidade de trabalhar com uma representação inferior.

Esta foi então gerada por um algoritmo rudimentar, que toma amostras de blocos de pixels de tamanho 4x4 da imagem, e cria um novo pixel cujos valores é a média dos valores dos píxels do bloco original. Assim, foi criada uma representação de dimensões (56, 56) para cada imagem, e a matriz bidimensional gerada ao aplanar as imagens ficou com 16 vezes menos atributos (9408) que a original.

Nesta manipulação, 98\% da variância é preservada. Porém, não é garantido que aspectos que podem ser fundamentais para o aprendizado de máquina foram preservados. Mesmo assim, por conta da restrição de tempo, ferramentas e qualificação, este caminho foi escolhido.

\section{Métodos}

Como foi dito, a aplicação do algoritmo de redução de dimensionalidade PCA falhou, sem conseguir terminar a execução, mesmo na matriz final de dimensões (2697, 9408). Isto é um forte indicativo de que haverão sérias restrições sobre a viabilidade dos métodos a serem utilizados.

Todavia, explorações mais profundas não foram feitas, pois foge do escopo da fase um do projeto.

\section{Código fonte produzido}

Como resultado do trabalho desta fase, foram escritas uma série de funções em python3 para carregamento de dados. Todas elas estão no diretório "pysrc/loading". Notavelmente, a função \textbf{loadImgDataset} está encarregada de resumir todo o trabalho, produzindo uma matriz em que cada linha se refere a uma imagem já com a qualidade reduzida pelo método mencionado na Seção~\ref{sec:manip}.

\clearpage

\end{document}
